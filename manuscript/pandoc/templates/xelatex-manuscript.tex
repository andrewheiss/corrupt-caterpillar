%!TEX program = xelatex
% Adapted in part from Nicholas Reith's Journal Article Manuscript Template
% https://www.overleaf.com/latex/templates/journal-article-manuscript-template/yhxfrgxqdthx
\documentclass[letterpaper,12pt]{article}

% ---------------
% GENERAL SETUP
% ---------------
\usepackage[american]{babel}
\usepackage[babel]{csquotes}
\usepackage{enumitem}
\usepackage{etoolbox}
\setlength{\parindent}{0.5in}


% -------
% FONTS
% -------
% mathtools has \underbracket and \overbracket with options for controlling 
% height and thickness, but unicode-math provides its own option-free versions 
% of these bracket macros. At one point it used to patch these macros and not 
% override mathtools' commands, but it currently does not do that anymore 
% (see https://github.com/wspr/unicode-math/issues/544). 
%
% We use unicode-math here so that we can use custom math fonts with 
% \setmathfont{}, which then messes up bracket options. But there's a solution! 
% We save mathtools' brackets  after loading mathtools, then load 
% unicode-math which clobbers them, and then *after* \begin{document}, redifine 
% the stored brackets (this part is key! According to 
% https://tug.org/pipermail/lualatex-dev/2011-August/001295.html, the unicode 
% table from unicode-math is executed after beginning the document, so 
% redefining the brackets in the preamble won't work)

% Math stuff
\usepackage{amsmath, amssymb, amsfonts, amsthm, mathtools}

% Save mathtools' brackets
\let\normalunderbracket=\underbracket
\let\normaloverbracket=\overbracket

\usepackage{unicode-math}  % For custom math fonts

% Custom fonts
\usepackage{fontspec}
\usepackage{xunicode}
\defaultfontfeatures{Mapping=tex-text,Ligatures=TeX,Scale=MatchLowercase}
$if(ms_mainfont)$
\setmainfont[Numbers={Proportional,OldStyle}]{$ms_mainfont$}
$else$
\setmainfont[Numbers={Proportional,OldStyle}]{Linux Libertine O}
$endif$
$if(ms_sansfont)$
\setsansfont{$ms_sansfont$}
$else$
\setsansfont{Source Sans Pro}
$endif$
$if(ms_monofont)$
\setmonofont[Mapping=tex-ansi, Scale=MatchLowercase]{$ms_monofont$}
$else$
\setmonofont[Mapping=tex-ansi, Scale=MatchLowercase]{InconsolataGo}
$endif$
$if(ms_mathfont)$
\setmathfont{$ms_mathfont$}
$else$
\setmathfont{Libertinus Math}
$endif$


% ---------------
% TITLE SECTION
% ---------------
% Abstract
\usepackage{abstract}
\renewcommand{\abstractnamefont}{\normalfont\normalsize\bfseries}
\renewcommand{\abstracttextfont}{\normalsize}
\setlength{\absleftindent}{\parindent}
\setlength{\absrightindent}{\parindent}

% Title
\usepackage{titlesec} % Allows customization of titles

% Authors
\usepackage{authblk} % For multiple authors

% Keywords
\providecommand{\keywords}[1]{\textbf{\textit{Keywords---}}#1}


% -------------------
% PAGE LAYOUT SUTFF
% -------------------
% For landscape PDF pages
\usepackage{pdflscape}

% For better TOCs
\usepackage{tocloft}

% Page margins
\usepackage[top=1in, bottom=1in, left=1in, right=1in]{geometry}

% Ensure that figures are floated where or after they're defined
\usepackage{flafter}

% Running header stuff
\usepackage{fancyhdr}
\setlength{\headheight}{0.25in}
\renewcommand{\headrulewidth}{0pt}  % Remove lines
\renewcommand{\footrulewidth}{0pt}

% SHORT TITLE           Page #
\fancypagestyle{normal}{
  \fancyhf{}
  \lhead{\uppercase{$if(short-title)$ $short-title$ $else$ $title$ $endif$}}
  \rhead{\thepage}
}

% Running head: SHORT TITLE        Page #
\fancypagestyle{title}{
  \fancyhf{}
  \lhead{Running head: \uppercase{$if(short-title)$ $short-title$ $else$ $title$ $endif$}}
  \rhead{}
}

% Use regular heading style
\pagestyle{normal}


% ------------------
% TYPOGRAPHY STUFF
% ------------------
% Pandoc tightlists
\providecommand{\tightlist}{%
  \setlength{\itemsep}{0pt}\setlength{\parskip}{0pt}}

% Fix widows and orphans
\usepackage[all,defaultlines=2]{nowidow}

% Don't typeset URLs in a monospaced font
\usepackage{url}
\urlstyle{same}
% Add - to url's default breaks
\def\UrlBreaks{\do\.\do\@\do\\\do\/\do\!\do\_\do\|\do\;\do\>\do\]%
    \do\)\do\,\do\?\do\&\do\'\do+\do\=\do\#\do-}

% Wrap definition list terms
% https://tex.stackexchange.com/a/9763/11851
\setlist[description]{style=unboxed}

% Format section headings
\titleformat*{\section}{\center\large\bfseries}
\titleformat*{\subsection}{\large\bfseries}
\titleformat*{\subsubsection}{\normalsize\bfseries}

% Disable section numbering
\setcounter{secnumdepth}{0}

% Fancier verbatims
\usepackage{fancyvrb}

% Line spacing
\usepackage{setspace}

% Single spacing in code chunks
\AtBeginEnvironment{verbatim}{\singlespacing}

% Hyperlink stuff
\usepackage[xetex, colorlinks=true, urlcolor=DarkSlateBlue,
            citecolor=DarkSlateBlue, filecolor=DarkSlateBlue, plainpages=false,
            hyperfootnotes=false,  % For compatibility with footmisc
            pdfpagelabels, bookmarksnumbered]{hyperref}

% Color names for hyperlinks
\usepackage[svgnames]{xcolor}

% Epigraphs
$if(epigraph)$
\usepackage{epigraph}
\renewcommand{\epigraphsize}{\footnotesize}
\setlength{\epigraphrule}{0pt}
$endif$

% Syntax highlighting stuff
$if(highlighting-macros)$
$highlighting-macros$
$endif$


% --------
% MACROS
% --------
\newcommand{\blandscape}{\begin{landscape}}
\newcommand{\elandscape}{\end{landscape}}
\newcommand{\stgroup}{\begingroup}
\newcommand{\fingroup}{\endgroup}

\newcommand{\memSingle}{}
\newcommand{\memSingleSmall}{}


% --------
% TABLES
% --------
\usepackage{booktabs}
\usepackage{longtable}
\usepackage{pbox}  % For multi-line table cells

% Things that kableExtra needs
\usepackage{array}
\usepackage{multirow}
\usepackage{wrapfig}
\usepackage{float}
\usepackage{colortbl}
\usepackage{tabu}
\usepackage{threeparttable}
\usepackage{threeparttablex}
\usepackage[normalem]{ulem}
\usepackage{makecell}

% Make table text slightly smaller, but not captions
\usepackage[font=normalsize]{caption}
\AtBeginEnvironment{longtable}{\footnotesize}
\AtBeginEnvironment{tabular}{\footnotesize}

% Remove left margin in lists inside longtables
% https://tex.stackexchange.com/a/378190/11851
\AtBeginEnvironment{longtable}{\setlist[itemize]{nosep, wide=0pt, leftmargin=*, before=\vspace*{-\baselineskip}, after=\vspace*{-\baselineskip}}}


% ----------
% GRAPHICS
% ----------
\usepackage{graphicx}
\makeatletter
\def\maxwidth{\ifdim\Gin@nat@width>\linewidth\linewidth\else\Gin@nat@width\fi}
\def\maxheight{\ifdim\Gin@nat@height>\textheight\textheight\else\Gin@nat@height\fi}
\makeatother
% Scale images if necessary, so that they will not overflow the page
% margins by default, and it is still possible to overwrite the defaults
% using explicit options in \includegraphics[width, height, ...]{}
\setkeys{Gin}{width=\maxwidth,height=\maxheight,keepaspectratio}


% ------------
% REFERENCES
% ------------
% NB: Using "strict" messes with footnotes and removes the rule!
$if(biblatex)$

$if(bibstyle-chicago-notes)$
\usepackage[notes, backend=biber,
            autolang=hyphen, bibencoding=inputenc,
            isbn=false, uniquename=false]{biblatex-chicago} % biblatex setup
$endif$
$if(bibstyle-chicago-authordate)$
\usepackage[authordate, backend=biber, noibid,
            autolang=hyphen, bibencoding=inputenc,
            isbn=false, uniquename=false]{biblatex-chicago} % biblatex setup
$endif$
$if(bibstyle-apa)$
\usepackage[style=apa, backend=biber]{biblatex}
\DeclareLanguageMapping{american}{american-apa}
$endif$

% No space between bib entries
\setlength\bibitemsep{0pt}

%% Fix biblatex's odd preference for using In: by default.
\renewbibmacro{in:}{%
  \ifentrytype{article}{}{%
  \printtext{\bibstring{}\intitlepunct}}}

%% bibnamedash: with Minion Pro the three-emdash lines in the
%% bibliogrpaphy end up separated from one another, which is very
%% annoying. Replace them with a line of appropriate size and weight.
\renewcommand{\bibnamedash}{\rule[3.5pt]{3em}{0.5pt}}

\addbibresource{$bibliography$}
\setlength\bibhang{\parindent}
$endif$


% -----------
% FOOTNOTES
% -----------
% NB: footmisc has to come after setspace and biblatex because of conflicts
\usepackage[bottom, flushmargin]{footmisc}
\renewcommand*{\footnotelayout}{\footnotesize}

\addtolength{\skip\footins}{12pt}    % vertical space between rule and main text
\setlength{\footnotesep}{16pt}  % vertical space between footnotes


% -----------------------
% ENDFLOAT AND ENDNOTES
% -----------------------
$if(endfloat)$
\usepackage[nolists]{endfloat}
\DeclareDelayedFloatFlavour*{longtable}{table}
$endif$

$if(endnotes)$
\usepackage{endnotes}
\renewcommand{\enotesize}{\normalsize}
\let\footnote=\endnote
$endif$


% -------------------
% STUFF FROM PANDOC
% -------------------
$for(header-includes)$
$header-includes$
$endfor$


% ------------------------------------------------------------------------------
% ------------------------------------------------------------------------------
% ------------------------------------------------------------------------------

% ----------------
% ACTUAL DOCUMENT
% ----------------

\begin{document}

% Restore mathtools' brackets *after* \begin{document}
\let\underbracket=\normalunderbracket
\let\overbracket=\normaloverbracket

$for(include-before)$
$include-before$
$endfor$


% ------------
% TITLE PAGE
% ------------
$if(title-page)$

\begin{titlepage}
\thispagestyle{title}
\center

\vspace*{0.5in}
{\large\bfseries $title$}

\vspace{\baselineskip}

$for(author)$
$author.name$ \\
$author.affiliation$ \\
$if(author.twitter)$@$author.twitter$ \\$endif$

$sep$ \vspace{\baselineskip}
$endfor$

\vspace{0.25in}

$if(word-count)$
{\center Word count: $word-count$}
$endif$

\vspace{0.5in}

$if(author-note)$
{\center Author note}

\begin{flushleft}
$author-note$
\end{flushleft}
$endif$

$if(correspondence)$
{\center Correspondence}

\begin{flushleft}
$correspondence$
\end{flushleft}
$endif$

$if(additional-info)$
{\center Additional information}

\begin{flushleft}
$additional-info$
\end{flushleft}
$endif$

$if(thanks)$
{\center Acknowledgments}

\begin{flushleft}
$thanks$
\end{flushleft}
$endif$

\vfill  % Fill the rest of the page with whitespace

\end{titlepage}

\doublespacing

\begin{abstract}
\noindent $abstract$
\end{abstract}

\vspace{\baselineskip}

$if(keywords)$\indent \keywords{$keywords$}$endif$

\newpage

$endif$


% ------------------------
% ACTUAL ACTUAL DOCUMENT
% ------------------------

\begin{center}
{\large\bfseries $title$}
\end{center}

$if(toc)$
{
\hypersetup{linkcolor=black}
\setcounter{tocdepth}{$toc-depth$}
\tableofcontents
$if(toc-figures)$
\listoffigures
$endif$
$if(toc-tables)$
\listoftables
$endif$
}
$endif$

$if(epigraph)$
$for(epigraph)$
$if(epigraph.source)$
\epigraph{$epigraph.text$}{---$epigraph.source$}
$else$
\epigraph{$epigraph.text$}
$endif$
$endfor$
$endif$

\doublespacing

$body$

\clearpage
\singlespacing
\printbibliography[heading=bibliography$if(reference-section-title)$, title=$reference-section-title$$endif$]

$if(endnotes)$
\clearpage
\begingroup
\parindent 0pt
\parskip 1em
\theendnotes
\endgroup
$endif$

$for(include-after)$
$include-after$
$endfor$

\end{document}
